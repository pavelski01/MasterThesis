\documentclass{article}
\usepackage{blindtext}
\usepackage[T1]{fontenc}
\usepackage[utf8]{inputenc}

\renewcommand{\thesection}{\Roman{section}} 

\title{Element twórczy pracy}
\author{Paweł Strzeszkowski}
\date{\today}

\begin{document}

\maketitle

\section{Cel i tematyka pracy dyplomowej}

\begin{enumerate}
    \item Rozprawa naukowa ma na celu przeanalizowanie aktualnie i historycznie dostępnych sposobów użycia różnych technologii, służących do tworzenia rzeczywistości rozszerzonej.
    \item Napisanie silnika oprogramowania mapującego otoczenie.
    \item Aplikacja mobilna do generowania demonstracji, której celem będzie zaimplementowanie przykładowego rozwiązania pozwalającego na tworzenie interaktywnych prezentacji.
\end{enumerate}

\section{Element twórczy}

\begin{enumerate}
    \item Elementem twórczym mojej pracy, będzie stworzenie silnika mapującego otoczenie.
    \item Wygenerowania przykładowej ekspozycji zmieniającej się w czasie rzeczywistym oraz porównanie z innymi tego typu prezentacjami (literatura, materiały cyfrowe).
\end{enumerate}

\end{document}

